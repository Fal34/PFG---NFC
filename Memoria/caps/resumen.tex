\documentclass[PFC.tex]{subfiles}
\begin{document}
Cuando una persona envía un mensaje a un destinatario con información que considera comprometida o personal, busca realizarlo con el mínimo riesgo de que dicho mensaje llegue de forma íntegra al. Además, éste quiere que el canal sea seguro y que nadie más intercepte la información; y si ocurriese tal caso, que personas ajenas no sean capaces de interpretar el mensaje y usarlo en su contra de forma perjudicial -o simplemente no desea difundir la información a alguien que no sea el destinatario-. También es evidente la necesidad de sistemas que eviten de la mejor forma posible la suplantación de usuarios; apoyándose en la criptografía, la autentificación resulta indespensable.


## ??? ##
Es conocido que actualmente los gobiernos, pretenden controlar mediante el acceso a las comunicaciones bajo pretextos de seguridad utilizan macro-redes que trabajan de forma conjunta para capturar la mayor información posible.  La privacidad es un derecho y escudarse en la seguridad por interés común no debería considerarse lícito entrometerse sin fundamento alguno. La criptografía sirve como elemento de control y protección. Existiendo numerosas formas de proteger la seguridad, integridad y disponibilidad de \\
## ??? ##

Existe la consideración generalizada que aquellos elementos más seguros a la hora de identificar y autenticar a un usuario de un sistema son aquellos que implican el uso de parámetros biométricos. Por ejemplo, el algoritmo para el reconocimiento del iris patentado por el investigador de la universidad de Cambridge, John Daugman, se basa en el iris como elemento único e intransferible para cada persona[\ref{irisramli}]. De forma análoga se utilizan algoritmos basados en la estructura facial o huellas dactilares. Por otra parte, Manuel Lucena López, doctor en informática de la universidad de Jaén, asegura en su publicación [\ref{lucena}] que esta clase de `requerimientos biométricos` se pueden reducir a problemas de autenticación basada en dispositivos. Es decir, una tarjeta puede actuar con el mismo compromiso de seguridad que dichos elementos biológicos.\\

Al igual que con los sistemas criptográficos (criptosistemas), la implementación más segura suele ser la menos eficiente. En la actualidad, un mundo globalizado donde la competencia  hace de la optimización un objetivo en el que se invierten millones en capital. En el mundo del motor un incremento de la velocidad punta de 3km/h puede implicar reducir el tiempo en unas décimas vitales que podrían implicar la diferencia entre ser primero o ser segundo. Dentro de la criptología y la seguridad de comunicaciones y autenticación también está afectado por este hecho y no está exento de la voracidad por optimizar el trinomio de recursos, tiempo y resultados. Dentro de este contexto la tecnología tan asequible como NFC (`Near Field Comunication`), - dispositivos para la comunicación de información empleando radiofrecuencia de corto alcance- que ha conquistado numerosos ámbitos, será el soporte de estudio de este `Trabajo de Fin de Grado` (TFG a partir de ahora) como dispositivo de autenticación.\\

Como suele ser habitual, un dispositivo `barato` tiene ciertas desventajas. Respecto a la seguridad criptográfica con NFC el mayor inconveniente suele ser el tamaño de almacenamiento - inferior a 500 bits generalmente-. Este problema implica utilizar criptosistemas que puedan trabajar con tamaños reducidos de información sin comprometer la seguridad.\\

Veremos cómo la evolución en la criptografía es considerable desde que en la Segunda Guerra Mundial el proyecto ULTRA tratara de descifrar los mensajes del ejército alemán; quienes se encontraban a la vanguardia de la criptografía. A su vez, se implementará una aplicación para `smartphones` que, gracias a algorítmos avanzados de criptografía, permitirán a un usuario crear y utilizar un `tag` NFC como dispositivo de autenticación en un sistema ficticio.

\end{document}