\documentclass[../PFC.tex]{subfiles}
\begin{document}
Hace décadas comunicarse mediante un dispositivo que estuviera conectado a una red cableada requería una larga espera y aún así resultaba algo fantástico. Actualmente tenemos la posibilidad de realizar un gesto en cualquier lugar y ponernos en contacto con alguien a cientos o miles de kilómetros. En este sentido, las comunicaciones han evolucionado de una forma increíble. La seguridad en estas comunicaciones es primordial. Debido a esto, el uso de la criptografía es un elemento vital para salvaguardar la privacidad de los usuarios y la del contenido.
\*
\vspace{0.5515cm}
\\
La criptografía no se ha quedado atrás y durante el último siglo su progreso ha seguido un ascenso vertiginoso al igual que el de los sistemas de comunicación. Desde los métodos más primitivos basados en cambiar una letra por la anexa; hasta complejos sistemas criptológicos (criptosistemas) que aprovechan ciertas propiedades matemáticas para preservar niveles de seguridad elevados con el gasto de menos recursos (computacionales y de almacenamiento). Optimizar los recursos es esencial para ser competitivo; y para ello la metodología de criptografía basada en curvas elípticas reduce sensiblemente la cantidad de almacenamiento necesaria respecto a otros algoritmos. De la mano va la tecnología NFC (\textit{Near Field Communication}), la cual ha simplificado los dispositivos de comunicación en algo tan pequeño y barato que ha conquistado el planeta en forma de multitud de aplicaciones.
\*
\vspace{0.5515cm}
\\
En este Trabajo Fin de Grado (TFG a partir de ahora), propondremos una solución para elaborar sistemas seguros con dispositivos de comunicación de bajo coste y criptología avanzada. A su vez, se implementará una aplicación para \textit{smartphones} que, gracias a algoritmos avanzados de criptografía, permitirán a un usuario crear y utilizar un \textit{tag} NFC como dispositivo de autenticación de alta seguridad en un sistema ficticio.
\*
\vspace{0.5515cm}
\\
\end{document}