\documentclass[../PFC.tex]{subfiles}
\begin{document}

Cuando una persona envía un mensaje a un destinatario con información que considera comprometida o personal, pretende realizarlo con el mínimo riesgo de que dicho mensaje llegue de forma alterada, el contenido sea visualizado por terceras personas y la certeza de que llegue al destinatario deseado. Además, éste quiere que el canal sea seguro y que nadie más intercepte la información; y si ocurriese tal caso, que personas ajenas no sean capaces de interpretar el mensaje y usarlo en su contra de forma perjudicial -o simplemente no desea difundir la información a alguien que no sea el destinatario-. También es evidente la necesidad de sistemas que eviten de la mejor forma posible la suplantación de usuarios; por ello, la autentificación apoyada en la criptografía resulta indespensable.
\*
\vspace{0.5515cm}
\\
Existe el consenso generalizado que aquellos elementos más seguros a la hora de identificar y autenticar a un usuario de un sistema son aquellos que implican el uso de parámetros biométricos. Por ejemplo, el algoritmo para el reconocimiento del iris patentado por el investigador de la universidad de Cambridge, John Daugman, se basa en el iris como elemento único e intransferible para cada persona\cite{ramli2008iris}. De forma análoga se utilizan algoritmos basados en la estructura facial o huellas dactilares. Por otra parte, Manuel Lucena López, doctor en informática de la universidad de Jaén, asegura en \cite{lucena} que esta clase de requerimientos biométricos se pueden reducir a problemas de autenticación basada en dispositivos. Es decir, una tarjeta puede actuar con el mismo compromiso de seguridad que dichos elementos biológicos.
\*
\vspace{0.5515cm}
\\
Al igual, en los sistemas criptográficos (criptosistemas), la implementación más segura suele ser la menos eficiente. En la actualidad, la competencia entre empresas e investigadores hace de la optimización un objetivo en el que se invierten ingentes cantidades de recursos. En el campo de la automoción competitiva, un incremento de la velocidad punta de 3km/h puede implicar reducir el tiempo de un competidor en unas décimas vitales que podrían suponer la diferencia entre ser primero o ser segundo. La criptología y la seguridad de comunicaciones también están afectadas por este hecho y no están exentas de la voracidad por optimizar el trinomio de recursos, tiempo y resultados. Dentro de este contexto de competencia e innovación, una tecnología tan asequible como NFC (\textit{Near Field Comunication}), - dispositivos para la comunicación de información empleando radiofrecuencia de corto alcance- que ha conquistado numerosos ámbitos como tarjetas monedero o tarjetas bancarias, será el soporte de estudio de este Trabajo de Fin de Grado (TFG a partir de ahora) como dispositivo de autenticación.
\*
\vspace{0.5515cm}
\\
Como suele ser habitual, un dispositivo barato tiene ciertas desventajas. Respecto a la seguridad criptográfica con NFC el mayor inconveniente suele ser el tamaño de almacenamiento - inferior a 500 bits generalmente-. Este problema implica utilizar criptosistemas que puedan trabajar con tamaños reducidos de información, sin comprometer la seguridad en ningún caso.
\*
\vspace{0.5515cm}
\\
La evolución en la criptografía es considerable desde que en la Segunda Guerra Mundial el proyecto ULTRA tratara de descifrar los mensajes del ejército alemán; quienes se encontraban a la vanguardia de la criptografía. Con el paso del tiempo, se avanzó hacia criptosistemas más complejos. En 1976 Whitfield Diffie y Martin Hellman publicaban un algoritmo para compartir claves seguras (llamado Diffie-Hellman)\cite{diffie1976new}, lo cuál sentó las bases del conocido algoritmo RSA \cite{rivest1978method} para encriptación asimétrica un año después. En 1985 Victor Miller y Neal Koblitz propusieron la utilización de curvas elípticas la encriptación en comunicaciones. Su estructura algebraica es notoriamente más compleja que la mayoría de los criptosistemas anteriores a esta. Sin embargo, su implementación la posiciona entre las más eficientes; capaces de conseguir claves más cortas con el mismo nivel de seguridad\cite{lucena}. Por ello, se ha implementado esta tecnología en multitud de escenarios: desde monedas virtuales como \textit{BitCoin} hasta el cifrado de \textit{Whatsapp}\cite{whatsappEncryption} trabajan con las propiedades de la criptografía con curvas elípticas.
\*
\vspace{0.5515cm}
\\
Finalmente, se verá la potencia de estos sistemas los cuales, utilizando chips NFC de capacidad reducida, pueden generar un sistema de autenticación seguro aplicable a en escenario real.

\section{Objetivo}
\label{Objetivo}

El principal objetivo de este TFG es constatar la capacidad de la criptografía con curvas elípticas de elaborar aplicaciones seguras con dispositivos NFC. Gracias a las propiedades de este tipo de criptosistemas, basados en curvas elípticas, se obtienen implementaciones computacionalmente seguras reduciendo considerablemente el tamaño de las claves a utilizar; ideal para chips NFC los cuales poseen ciertas limitaciones en cuanto a capacidad de almacenamiento. De la mano de la criptografía y la tecnología NFC, se estudiará cuán seguro puede ser una aplicación que utilice estas bases junto a la realización de una aplicación de ejemplo.

\end{document}