\documentclass[../PFC.tex]{subfiles}
\begin{document}

\section{Introducción}
\label{App:Introducción}

Anteriormente se han explicado los conocimientos y términos generales relacionados con la seguridad en la autenticidad NFC basada en criptología con curvas elípticas. A continuación se muestra la aplicación de dichos conocimientos en un proyecto software experimental. El objetivo es demostrar la capacidad de los dispositivos NFC para, de la mano de la criptografía con curvas elípticas, llegar a desarrollar un sistema de autenticación óptimo y fiable.\\

Tanto los nombres, librerías, herramientas, así como el resto del material utilizado se describirán a continuación. A su vez, siguiendo un estándar de desarrollo software basado en metodologías ágiles, se mostrará la utilizada para éste proyecto. Para ello,  el autor y director de este TFG (Fidel Abascal y Domingo Gómez) hemos actuado y ejercido tanto de cliente como de contratado para el desarrollo de la aplicación. \\

Dentro de un ámbito ficticio, una empresa llamada \textbf{Alpha - Consultora S.A.}\footnote{Cualquier similitud con la realidad es mera coincidencia}, nueva potencia local dentro del campo de la seguridad bancaria, que ha cosechado unos excelentes resultados a lo largo de sus 2 años de existencia. Cuenta con más de 40 trabajadores y su crecimiento y expansión es notoria. Tanto es el éxito de esta empresa que, para dar cabida a su plantilla, ha decidido trasladarse a una nueva sede más moderna, amplia y mejor ubicada. La empresa, antes de instalarse en la nueva sede, decide contratar a unos expertos en seguridad para gestionar el control de accesos mediante un sistema de tarjetas y lectores en las entradas; teniendo en cuenta una inversión mínima pero garantizando un alto grado de seguridad. \\

Tras buscar incesantemente recurren a la empresa \textbf{F-NFC}. Una vez realizado el estudio por parte de F-NFC, se pone en consonancia un acuerdo para elaborar una aplicación que genere información que autentifique a un usuario de la empresa \textit{Alpha} y sea reconocido de forma unívoca para permitir su acceso a la sede. La infraestructura de la empresa propietaria de la nueva sede corresponde a la figura \ref{img:infraestructura}.

INSERTAR IMAGEN DEL SERVIDOR-TARJETA IDENTIFICATIVA-LECTOR
\label{img:infraestructura}

Los detalles de implementación de la aplicación seguirán un objetivo didáctico y experimental que cumplirán los requerimientos de la situación propuesta; adecuándose a la carencia real de la infraestructura anteriormente mencionada. Se explicará en las secciones siguientes en detalle todos los puntos implicados en la consecución de este objetivo.

\section{Material y metodología utilizada}
\label{Material y metodología utilizada}


\end{document}