\documentclass[../PFC.tex]{subfiles}
\begin{document}

% Versión en inglés %

Hace décadas comunicarse mediante un dispositivo que estuviera conectado a una red cableada y tras mucha espera resultaba algo fantástico. Hoy en día tenemos la posibilidad de realizar un gesto en cualquier lugar y ponernos en contacto con alguien a cientos o miles de kilómetros. En este sentido, las comunicaciones han evolucionado de una forma increíble. La seguridad en estas comunicaciones basada en la criptografía es un elemento primordial para salvaguardar la privacidad de los usuarios y la del contenido.\\

La criptografía no se ha quedado atrás y durante el último siglo su devenir ha seguido el mismo camino. Desde los métodos más primitivos basados en cambiar una letra por la anexa; hasta complejos sistemas criptológicos (criptosistemas) que aprovechan ciertas propiedades matemáticas para preservar niveles de seguridad elevados con la necesidad de menos recursos (computacionales y de almacenamiento). Optimizar los recursos es esencial para ser competitivo y para ello la metodología de criptografía basada en curvas elípticas reduce exponencialmente la cantidad de almacenamiento necesaria respecto a otros algoritmos. De la mano va la tecnología NFC (\textit{Near Field Comunication}), la cual ha simplificado los dispositivos de comunicación a algo tan pequeño y barato que ha conquistado el planeta en forma de multitud de aplicaciones.\\

En este Trabajo Fin de Grado (TFG a partir de ahora), comprenderemos la posibilidad de elaborar sistemas seguros con dispositivos de comunicación de bajo coste y criptología avanzada. A su vez, se implementará una aplicación para \textit{smartphones} que, gracias a algorítmos avanzados de criptografía, permitirán a un usuario crear y utilizar un \textit{tag} NFC como dispositivo de autenticación de alta seguridad en un sistema ficticio.

\end{document}