\documentclass[../PFC.tex]{subfiles}
\begin{document}

En este TFG se han podido observar las características de la encriptación con curvas elípticas junto a tarjetas NFC, así como su potencia, complejidad y versatilidad. También se ha visto la efectividad de estos criptosistemas para aplicaciones como la autenticación del personal de una empresa o la seguridad de comunicaciones. Todo esto nos hace ver que el avance en las tecnologías de la comunicación va ligado al avance de los métodos de seguridad que los protegen. Visto con perspectiva, las investigaciones y futuros desarrollos tratan de impulsar una civilización globalizada basada en la comunicación.

\section{Futuras mejoras}
\label{Futuras mejoras}

Dentro de las posibles mejoras e implementaciones futuras de la aplicación desarrollada se encuentra la ya comentada en el apartado \ref{App:Materiales y tecnologías utilizadas} de utilizar tarjetas \textit{MIFARE Classic 1k o 4k}. De esta forma se utilizarían chips que permitieran un extra de seguridad para hacer de la aplicación un sistema de seguridad perfectamente válido y utilizable en un escenario real.
\*
\vspace{0.5515cm}
\\
A su vez, en base a la autenticación por NFC, se podría elaborar la posibilidad de crear comunicaciones entre los distintos usuarios de la aplicación. Esto es, poder realizar comunicaciones seguras entre diferentes usuarios. Para ello sería idóneo implementar una comunicación basada en identificación con curvas elípticas; tal y como incitan a emplear en\cite{boneh2001identity} con \textit{Pairing}\cite{dupont2006provably} (ver Anexo B \ref{AnexoII}).
\*
\vspace{0.5515cm}
\\
Finalmente, se pueden contar con una multitud de aplicaciones posibles para el desarrollo realizado. Todo aquello que requiera identificación segura, firma digital o certificación entre otros hace de lo expuesto en este TFG una base con la cuál continuar.

\end{document}