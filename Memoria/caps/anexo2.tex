\documentclass[../PFC.tex]{subfiles}
\begin{document}

Gracias a \textit{Sagemath}\cite{sagemath} se puede realizar una implementación de ejemplo para el uso de pairings. El código se mostrado a continuación se ha ejecutado dentro de un terminal de la propia aplicación online.

\begin{lstlisting}
> Se crea una curva eliptica con las definiciones dadas
> p = 103; A = 1; B = 18; E = EllipticCurve(GF(p), [A, B])
> Se seleciona un punto de la curva y se muestra el orden del punto elegido
> P = E(33, 91); n = P.order(); n
19
> Se obtiene el orden multiplicativo
> k = GF(n)(p).multiplicative_order(); k
6
> Se calcula el tate pairing con P, n y k
> P.tate_pairing(P, n, k)
1
> Se calcula 3*P
> 3*P
(8 : 34 : 1)
> Se muestran los puntos de la curva
> for point in E:
>    print point
(0 : 1 : 0)
(0 : 11 : 1)
(0 : 92 : 1)
(2 : 50 : 1)
(2 : 53 : 1)
(6 : 31 : 1)
(6 : 72 : 1)
(7 : 33 : 1)
(7 : 70 : 1)
(8 : 34 : 1)
(8 : 69 : 1)
(12 : 25 : 1)
(12 : 78 : 1)
(14 : 43 : 1)
(14 : 60 : 1)
(15 : 3 : 1)
(15 : 100 : 1)
(17 : 2 : 1)
(17 : 101 : 1)
(18 : 10 : 1)
(18 : 93 : 1)
(19 : 43 : 1)
(19 : 60 : 1)
(20 : 2 : 1)
(20 : 101 : 1)
(21 : 37 : 1)
(21 : 66 : 1)
(22 : 39 : 1)
(22 : 64 : 1)
(24 : 8 : 1)
(24 : 95 : 1)
(26 : 25 : 1)
(26 : 78 : 1)
(27 : 40 : 1)
(27 : 63 : 1)
(28 : 33 : 1)
(28 : 70 : 1)
(29 : 5 : 1)
(29 : 98 : 1)
(32 : 8 : 1)
(32 : 95 : 1)
(33 : 12 : 1)
(33 : 91 : 1)
(37 : 49 : 1)
(37 : 54 : 1)
(38 : 21 : 1)
(38 : 82 : 1)
(42 : 20 : 1)
(42 : 83 : 1)
(43 : 19 : 1)
(43 : 84 : 1)
(45 : 41 : 1)
(45 : 62 : 1)
(47 : 8 : 1)
(47 : 95 : 1)
(48 : 6 : 1)
(48 : 97 : 1)
(50 : 51 : 1)
(50 : 52 : 1)
(51 : 46 : 1)
(51 : 57 : 1)
(52 : 17 : 1)
(52 : 86 : 1)
(54 : 7 : 1)
(54 : 96 : 1)
(55 : 0 : 1)
(62 : 13 : 1)
(62 : 90 : 1)
(64 : 20 : 1)
(64 : 83 : 1)
(65 : 25 : 1)
(65 : 78 : 1)
(66 : 2 : 1)
(66 : 101 : 1)
(68 : 33 : 1)
(68 : 70 : 1)
(69 : 51 : 1)
(69 : 52 : 1)
(70 : 43 : 1)
(70 : 60 : 1)
(72 : 13 : 1)
(72 : 90 : 1)
(77 : 21 : 1)
(77 : 82 : 1)
(81 : 36 : 1)
(81 : 67 : 1)
(83 : 49 : 1)
(83 : 54 : 1)
(84 : 12 : 1)
(84 : 91 : 1)
(86 : 49 : 1)
(86 : 54 : 1)
(87 : 51 : 1)
(87 : 52 : 1)
(89 : 12 : 1)
(89 : 91 : 1)
(91 : 21 : 1)
(91 : 82 : 1)
(92 : 18 : 1)
(92 : 85 : 1)
(93 : 48 : 1)
(93 : 55 : 1)
(94 : 1 : 1)
(94 : 102 : 1)
(95 : 42 : 1)
(95 : 61 : 1)
(97 : 38 : 1)
(97 : 65 : 1)
(100 : 20 : 1)
(100 : 83 : 1)
(101 : 27 : 1)
(101 : 76 : 1)
(102 : 4 : 1)
(102 : 99 : 1)
\end{lstlisting}

\end{document}