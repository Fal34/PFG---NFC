\documentclass[a4paper,12pt, twoside, openright,makeidx]{book}
%\usepackage[a4paper,left=20mm,right=20mm,top=25mm,bottom=25mm,]{geometry}
\usepackage[a4paper,left=25mm,right=25mm,top=30mm,bottom=30mm,total={210mm,297mm}]{geometry}
%\usepackage[a4paper,total={210mm,297mm}]{geometry}



\usepackage{subfiles}
%\usepackage[utf8]{inputenc}
\usepackage[utf8]{inputenc}
\usepackage{lmodern}
\usepackage[T1]{fontenc}    %Interprete de tildes
\usepackage[english,spanish]{babel}

\usepackage{hyperref}
\usepackage{glossaries}		%Soporte para glosario

\usepackage{amsmath,amssymb}    %Paquete de entornos matematicos
\usepackage{mathtools}
\usepackage{graphicx}
\usepackage{psfrag}
\usepackage{quotchap}
\usepackage{epsfig}
\usepackage[all]{xy}


\usepackage{wasysym}	% XBox / CheckedBox characters

\usepackage{makeidx}	%indice
\usepackage{multicol,multirow}
\setcounter{tocdepth}{3}
\hypersetup{backref,
    colorlinks,%
    citecolor=blue,%
    filecolor=black,%
    linkcolor=[rgb]{0.058823529,0.239215686,0.466666667
},%
    urlcolor=cyan
    }

\usepackage[toc,page]{appendix}    % Apendices

% Bibliografia
\usepackage{cite}
% Comentarios
\newcommand{\comm}[1]{\marginpar{%
\vskip-\baselineskip %raise the marginpar a bit
\raggedright\footnotesize
\itshape\hrule\smallskip#1\par\smallskip\hrule}}

% Custom Environments
\newenvironment{abstract}%
{\cleardoublepage\null \vfill\begin{center}%
\bfseries \abstractname \end{center}}%
{\vfill\null}

% index
%\let\orgtheindex\theindex
%\let\orgendtheindex\endtheindex
%\def\theindex{%
%  \def\twocolumn{\begin{multicols}{2}}%
%  \def\onecolumn{}%
%  \clearpage
%  \orgtheindex
%}
%\def\endtheindex{%
%  \end{multicols}
%  \orgendtheindex
%}
% /index


% Página en blanco
\usepackage{afterpage}

\newcommand\blankpage{%
    \null
    \thispagestyle{empty}%
    \addtocounter{page}{-1}%
    \newpage}

\makeglossaries
\makeindex

%% Datos del PFC
\newcommand{\titulo}{Seguridad utilizando dispositivos NFC}
\newcommand{\autor}{Fidel Abascal López}
\newcommand{\director}{Domingo Gómez Pérez}
\newcommand{\tutor}{}
\newcommand{\vocal}{Nombre Apellido1 Apellido2}
\newcommand{\vocalsup}{Nombre Apellido1 Apellido2}
\newcommand{\presidente}{Nombre Apellido1 Apellido2}
\newcommand{\presidentesup}{Nombre Apellido1 Apellido2}
\newcommand{\carrera}{Grado en Ingeniería Informática}
\title{\titulo}
\author{\autor}

\begin{document}

%###### - Portada - ######
\begin{titlepage}

\begin{center}
\vspace*{-1in}
\begin{figure}[htb]
\begin{center}
\includegraphics[width=5cm]{./img/unicanLogo}
\end{center}
\end{figure}

FACULTAD DE CIENCIAS\\
\vspace*{0.15in}
DEPARTAMENTO DE... \\
\vspace*{0.6in}
\begin{large}
PHD THESIS:\\
\end{large}
\vspace*{0.2in}
\begin{Large}
\textbf{EL TÍTULO DE LA TESIS ES MUY IMPORTANTE, ASÍ QUE, NO OLVIDES PONER UNO QUE SEA INTERESANTE Y ADECUADO PARA TU TESIS} \\
\end{Large}
\vspace*{0.3in}
\begin{large}
A Thesis submitted by Amy Wong for the degree of Doctor of Philosophy in the Mars University\\
\end{large}
\vspace*{0.3in}
\rule{80mm}{0.1mm}\\
\vspace*{0.1in}
\begin{large}
  \begin{flushleft}
Supervised by: \\
Hubert J. Farnsworth \\
\end{flushleft}

\end{large}
\end{center}

\end{titlepage}

%###### - Fin Portada - ######
\tableofcontents{}
\listoffigures

\chapter{Resumen}
\subfile{./caps/resumen}

\chapter{Abstract}
\subfile{./caps/abstract}

\chapter{Introducción}
\subfile{./caps/introduccion}

% #### - Parte teórica y fundamentación - ####%


%\chapter{Motivación}
%\subfile{caps/motivacion}
%\chapter{Objetivo}
%\subfile{caps/objetivo}
%\chapter{Criptología}
%\subfile{caps/criptologia}
%\chapter{Seguridad}
%\subfile{caps/seguridad}
%\chapter{Protocolos simétricos y asimétricos}
%\subfile{caps/protocolos}
%\chapter{Logaritmos discretos}
%\subfile{caps/logaritmos}
%\chapter{Criptología con curvas elípticas - ECC}
%\subfile{caps/criptografiaECC}
%\chapter{Tecnología NFC}
%\subfile{caps/nfc}
%
%% #### - Desarrollo de la aplicación - #### %
%\chapter{Autenticación en sistema ficticio}
%\subfile{caps/aplicacion}
%
%% ### - Conclusiones y futuras mejoras - ### %
%\chapter{Conclusiones}
%\subfile{caps/conclusiones}

% #### - Bibliografía & Referencias -  ###
\bibliographystyle{plain}
\bibliography{sources}
\nocite{*}
\afterpage{\blankpage}	% Página en blanco

% ### - Fin Referencias - ###

\end{document}
%sagemathcloud={"latex_command":"pdflatex -synctex=1 -interact=nonstopmode 'PFC.tex'"}
