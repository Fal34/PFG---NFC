\documentclass[a4paper,12pt, twoside, openright,makeidx]{book}
%\usepackage[a4paper,left=20mm,right=20mm,top=25mm,bottom=25mm,]{geometry}
\usepackage[a4paper,left=25mm,right=25mm,top=30mm,bottom=30mm,total={210mm,297mm}]{geometry}
%\usepackage[a4paper,total={210mm,297mm}]{geometry}

% Definicion para códigos %
\usepackage{listings}
\usepackage{color}

\definecolor{dkgreen}{rgb}{0,0.6,0}
\definecolor{gray}{rgb}{0.5,0.5,0.5}
\definecolor{mauve}{rgb}{0.58,0,0.82}

\lstset{frame=tb,
  language=Java,
  aboveskip=2mm,
  belowskip=2mm,
  showstringspaces=false,
  columns=flexible,
  basicstyle={\small\ttfamily},
  numbers=none,
  numberstyle=\tiny\color{gray},
  keywordstyle=\color{blue},
  commentstyle=\color{dkgreen},
  stringstyle=\color{mauve},
  breaklines=true,
  breakatwhitespace=true,
  tabsize=2
}

\usepackage{subfiles}
\usepackage[utf8]{inputenc}
\usepackage{lmodern}
\usepackage[T1]{fontenc}    %Interprete de tildes
\usepackage[english,spanish]{babel}

\usepackage{hyperref}
\usepackage{glossaries}		%Soporte para glosario

\usepackage{amsmath,amssymb}    %Paquete de entornos matematicos
\usepackage{mathtools}
\usepackage{graphicx}
\usepackage{psfrag}
\usepackage{quotchap}
\usepackage{epsfig}
\usepackage[all]{xy}
\usepackage{float}
%\usepackage{subfig}
\usepackage{caption}
\usepackage{subcaption} % for 'subfigure' environment
\usepackage{usecases} % Casos de uso

\usepackage{wasysym}	% XBox / CheckedBox characters
\usepackage{makeidx}	%indice
\usepackage{multicol,multirow}
\setcounter{tocdepth}{3}
\hypersetup{backref,
    colorlinks,%
    citecolor=blue,%
    filecolor=black,%
    linkcolor=[rgb]{0.058823529,0.239215686,0.466666667
},%
    urlcolor=cyan
    }

\usepackage[toc,page]{appendix}    % Apendices

% Bibliografia
\usepackage{cite}
% Comentarios
\newcommand{\comm}[1]{\marginpar{%
\vskip-\baselineskip %raise the marginpar a bit
\raggedright\footnotesize
\itshape\hrule\smallskip#1\par\smallskip\hrule}}

% Custom Environments
\newenvironment{abstract}%
{\cleardoublepage\null \vfill\begin{center}%
\bfseries \abstractname \end{center}}%
{\vfill\null}

% index
%\let\orgtheindex\theindex
%\let\orgendtheindex\endtheindex
%\def\theindex{%
%  \def\twocolumn{\begin{multicols}{2}}%
%  \def\onecolumn{}%
%  \clearpage
%  \orgtheindex
%}
%\def\endtheindex{%
%  \end{multicols}
%  \orgendtheindex
%}
% /index


% Página en blanco
\usepackage{afterpage}

\newcommand\blankpage{%
    \null
    \thispagestyle{empty}%
    \addtocounter{page}{-1}%
    \newpage}

\makeglossaries
\makeindex

%% Datos del PFC
\newcommand{\titulo}{Seguridad utilizando dispositivos NFC}
\newcommand{\tituloIngles}{Security using NFC devices}
\newcommand{\autor}{Fidel Abascal López}
\newcommand{\director}{Domingo Gómez Pérez}
\newcommand{\tutor}{}
\newcommand{\carrera}{GRADO EN INGENIERÍA INFORMÁTICA}
\newcommand{\fecha}{Junio - 2016}
\renewcommand{\thefootnote}{\fnsymbol{footnote}}
\title{\titulo}
\author{\autor}

\begin{document}

%###### - Portada - ######
\begin{titlepage}

\begin{center}
\vspace*{-1in}
\begin{figure}[htb]
\begin{center}
\includegraphics[width=4cm]{./img/unicanLogo}
\end{center}
\end{figure}

\begin{LARGE}
\textbf{
Facultad \\
De\\
Ciencias\\}
\end{LARGE}
\vspace*{1in}
\begin{huge}
\textbf{\titulo} \\
\tituloIngles \\
\end{huge}
\vspace*{0.6in}
\begin{large}
Trabajo de Fin de Grado\\
para acceder al\\
\vspace*{0.15in}
\textbf{\carrera}\\
\end{large}
\vspace*{0.8in}

\begin{large}
\begin{flushright}
Autor : \autor\\
Director : \director\\
\fecha
\end{flushright}
\end{large}

\end{center}

\end{titlepage}
%###### - Fin Portada - ######
\begingroup
\let\cleardoublepage\clearpage
\tableofcontents{}
\endgroup

\begingroup
\let\cleardoublepage\clearpage
\listoffigures
\endgroup

\chapter*{Resumen}
\label{Resumen}
\subfile{./caps/resumen}

\label{Palabras clave}
\begin{center}
\textbf{Palabras clave}: \textit{Criptografía, Curvas elípticas, NFC, autenticación}.
\end{center}

\newpage
\label{Abstract}
\begin{center}
\textbf{Abstract}
\end{center}
\subfile{./caps/abstract}

\label{Keywords}
\begin{center}
\textbf{Keywords}: \textit{Cryptography, elliptic curves, NFC, authentication}.
\end{center}

\chapter{Introducción}
\label{Introducción}
\subfile{./caps/introduccion}

% #### - Parte teórica y fundamentación - ####%
\chapter{Elementos teóricos}
\label{Elementos teoricos}
\subfile{caps/teoria}
%
%% #### - Desarrollo de la aplicación - #### %
\chapter{APP: Autenticación en un sistema ficticio}
\label{APP: Autenticación en sistema ficticio}
\subfile{caps/aplicacion}
%
%% ### - Conclusiones y futuras mejoras - ### %
\chapter{Conclusiones}
\label{Conclusiones}
\subfile{caps/conclusiones}

%% ### - Anexo I - ###
\afterpage{\blankpage}	% Página en blanco
\chapter*{Anexo I: Comunicación de claves}
\label{AnexoI}
\subfile{caps/anexo1}

%% ### - Anexo II - ###
\chapter*{Anexo II: EC - Pairing}
\label{AnexoII}
\subfile{caps/anexo2}

% #### - Bibliografía & Referencias -  ###
\bibliographystyle{plain}
\bibliography{sources}
\nocite{*}
\afterpage{\blankpage}	% Página en blanco

% ### - Fin Referencias - ###

\end{document}
%sagemathcloud={"latex_command":"pdflatex -synctex=1 -interact=nonstopmode 'PFC.tex'"}
