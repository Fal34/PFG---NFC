\documentclass[PFC.tex]{subfiles}
\begin{document}
En la actualidad, las manifestaciones o las huelgas e incluso otro ámbito como el fanatismo futbolístico, están muy presentes. Tanto es así, que cada día somos conscientes de la cantidad de noticias en el periódico, televisión o redes sociales, que nos encontramos hablando del número de personas que han acudido a una manifestación o la multitud de aficionados viendo un partido de fútbol. Cualquier medio de comunicación te dice un número aproximado de personas que han “salido a la calle” o que han disfrutado en el campo de fútbol. Pero, ¿es cierto que había 2000 personas (aproximadamente) manifestándose aquel día o que ‘miles’ de aficionados disfrutaban de un Barça-Real Madrid aquella tarde? y ¿en qué se basan los medios de comunicación para dar esas cifras? Son preguntas que la mayoría de las veces nos hacemos, ya que los distintos medios de comunicación difieren en las cifras con gran diferencia entre unas y otras, en muchas ocasiones.\\

Sin embargo, es razonable entender lo difícil que es contar o saber cuántas personas hay en una manifestación o en un partido de fútbol, cuando el número es evidentemente muy grande. Si tuviéramos, por ejemplo, 25 personas, no sería díficil ‘contar con el dedo’ una por una hasta llegar a contar 25 personas. Pero, con cantidades muy grandes, el método del ‘conteo con el dedo’ se nos queda complicado y perderíamos mucho tiempo.\\

Por esta razón, se ha pensado en crear una aplicación para sistemas Android que permita al usuario estimar el número de personas que hay en una foto (por ejemplo,el de una manifestación). La manera de proceder, en primer lugar, ha sido implementar un algoritmo en Python (software) que nos permitiera estimar el número de personas en una fotografía, así como el error cometido en esa estimación, es decir, la varianza. Para que esta estimación resultara más sencilla, se ha dividido la imagen en cuadrados, mediante una rejilla o cuadrícula también implementada por el algoritmo. De esta manera, se consigue reducir la dificultad del problema, ya que se estimaría el número de partículas/personas en cada cuadrado.

\end{document}